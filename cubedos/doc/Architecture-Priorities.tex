
\section{Message Priorities}
\label{section-message-priorities}

In this section we describe the architecture of message priorities in CubedOS. Priorities are
built on top of the task priority system provided by the Ada runtime environment. CubedOS does
not directly interact with the underlying operating system's notion of priorities. This provides
portability to any system with a suitably capable Ada runtime environment. Specifically, message
priorities are supported on selected bare board systems without an operating system.

\subsection{Justification}

\pet{Why do we need this? Fill out this section when we know!}

\subsection{Architecture}

All modules are assigned a ``base priority'' statically using suitable Ada language pragmas
applied to the module's server task. We define the server task of a module as the task that
reads the module's mailbox and processes the module's messages. We speak of a module's priority
as a shorthand for the ``priority of a module's server task.'' Note that modules are allowed to
have internal tasks in addition to the main task. The priority of those tasks is an internal
matter that does not concern us here.

CubedOS also allows messages to be given priorities that can be distinct from the priorities of
the modules that send and receive them. Message priorities are not a concept known to the Ada
runtime environment; they are purely a construct of CubedOS.

When a module sends a message that message is, by default, given a priority equal to that of the
sending module. However, a module can optionally lower the priority of a message below its own
priority. If a module attempts to raise the priority of a message above its own priority, it is
not an error, either statically or dynamically, but the priority is simply set to the module's
priority instead. \pet{It might be better if this was a statically detected error.}

Messages are removed from a module's mailbox in priority order in $O(\log\,n)$ time (where $n$
is the number of pending messages in the mailbox). \pet{I'm assuming the mailboxes are priority
  heaps} When a module with priority $m$ is processing a message at priority $p > m$, the
module's priority is dynamically increased to $p$ and any messages it sends, including replies
to the original message, are sent by default at priority $p$ (although the module can downgrade
the priority of sent messages, as usual). When a module completes the processing of the high
priority message, just before it fetches the next message from the mailbox, its priority is
returned to its original base priority level.

When a module with priority $m$ is processing a message at priority $p < m$, the module's
priority remains unchanged at its base priority. Thus a high priority module services low
priority messages at its usual (high) priority. In contrast a low priority module services high
priority messages at a temporarily elevated priority. This \newterm{priority inheritance} is
needed so a high priority client isn't slowed down when it requests service from a low priority
module. When executing on behalf of a high priority client, a low priority server inherits the
priority of the client temporarily.

This structure also allows a high priority module to downgrade the priority of outgoing messages
when servicing a low priority client. The idea is that the high priority module might want to
make requests to other modules using the client's priority (which is available to it in the
original request message sent by the client). It is up the application developer to do this if
it is desired. The default behavior is for all sent messages to be given the priority of the
sender.

\subsection{Static Analysis}

Unfortunately \SPARK\ does not currently support the dynamic modification of task priorities.
Thus use of that feature in a CubedOS application must be specifically justified so as to avoid
spurious \SPARK\ diagnostics. However, doing this means that \SPARK\ can no longer be relied
upon to find all potential deadlocks or livelocks. It is therefor necessary to employ an
additional form of analysis that can supplement \SPARK\ and statically show freedom from those
issues in any case. This additional analysis has yet to be defined. It can, however, take
advantage of certain restrictions that still remain in CubedOS tasking despite the addition of
dynamic priorities.

\begin{itemize}
\item Modules only interact via message passing.
\item Messages carry priorities and module priorities are dynamically adjusted as described above.
\item No other dynamic priority adjustments are allowed. \pet{It might be necessary to forbid a
    module from lowering the priority of its outgoing messages\ldots\ perhaps in a first
    implementation anyway} This is statically checked with a supplementary tool. \pet{Using
    libadalang, perhaps?}
\end{itemize}

\pet{I wonder if spin can be used to do what we need. Perhaps a spin module can be automatically
  extracted from the Ada source using a supplementary tool. That same tool could perhaps do the
  additional static checking mentioned above}

